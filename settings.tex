%
% Datei:			Einstellungen.tex
% Ersteller: 	Daniel G�rges
% Datum:			15.02.2016
% Status:			Vorlage
%

\usepackage{eurosym}

%
% Einstellungen f�r die Klasse EXAM
%
\pointpoints{Punkt}{Points}
\qformat{\textbf{\large Problem \thequestion}\quad\textbf{\large(\thepoints)}\hfill\rule[-3ex]{0ex}{3ex}}

%
% AMS-Operatoren
%
\DeclareMathOperator{\sign}{sign}

%
% Punkte auf der Nebendiagonale
%
\newcommand{\rdots}{\mathinner{
  \mkern1mu\raise1pt\hbox{.}
  \mkern2mu\raise4pt\hbox{.}
  \mkern2mu\raise7pt\vbox{\kern7pt\hbox{.}}\mkern1mu}}

%
%Vektoren/Matizen fett
%
\renewcommand{\vec}[1]{\boldsymbol{#1}}

%erm�glicht fette griechische Buchstaben
\newcommand{\MB}[1]{{\mbox{\mathversion{bold}$#1$}}}

%
%Matrix (runde Klammer)
%
\newcommand{\mat}[1]{\begin{pmatrix}#1\end{pmatrix}}

%
%Darstellung von Hinweisen
%
\newcommand{\Hinweis}{\textbf{Hinweis:~}}
\newcommand{\Hint}{\textbf{Hint:~}}
\newcommand{\Annotation}{\textbf{Annotation:~}}

%
%Partielle Ableitung mit Potenz.
%Nutzung: \parfr[potenz(optional)]{z�hler}{nenner}
%
\newcommand{\parfr}[3][\empty]{
	\ifthenelse{\equal{#1}{\empty}}
	{\frac{\partial {#2}}{\partial {#3}}}
	{\frac{\partial^{#1} {#2}}{\partial {#3}^{#1}}}
	}
	
%
%Normale Ableitung mit Potenz
%Nutzung: \dfr[potenz(optional)]{z�hler}{nenner}
%
\newcommand{\dfr}[3][\empty]{
	\ifthenelse{\equal{#1}{\empty}}
	{\frac{\text{d} {#2}}{\text{d} {#3}}}
	{\frac{\text{d}^{#1} {#2}}{\text{d} {#3}^{#1}}}
	}

%
%Zahl in einem Kreis, geht im Text und auch in eine Formel
%
\newcommand{\mycirc}[1]{\mbox{\textcircled{\footnotesize #1}}}

%
%Bepunktung, geht im Text und auch in eine Formel
%
\newcommand{\Punkte}[1]{\text{~\textbf{($\boldsymbol{#1}$~P)}}}


\addto\captionsenglish{\renewcommand{\figurename}{Fig.}}

%
% Angaben zu der Pr�fung
%
\newcommand{\pruefungsname}{Principles of Electrical and Computer Engineering for Commericial Vehicle Technology}
\newcommand{\pruefungsnamekurz}{Principles of ECE for CVT}
\newcommand{\pruefungsnummer}{20552}
\newcommand{\pruefungsdatum}{22 March 2016}
\newcommand{\pruefungszeitraum}{WS 2015/2016}
\newcommand{\pruefungszettel}{1}