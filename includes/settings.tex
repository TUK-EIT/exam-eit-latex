%%%%%%%%%%%%%%%%%%%%%%%%%%%%%%%%%%%%%%%%%%%%%%%%%%%%%%%%%%%%%%%%%%%%%%%
%% Template Configuration
%%%%%%%%%%%%%%%%%%%%%%%%%%%%%%%%%%%%%%%%%%%%%%%%%%%%%%%%%%%%%%%%%%%%%%

% LaTeX exam class configuration
\iftoggle{isGerman}{
	\pointpoints{Punkt}{Punkte}
	\qformat{\textbf{\large Aufgabe \thequestion}\quad\textbf{\large(\thepoints)}\hfill\rule[-3ex]{0ex}{3ex}}
}{
	\pointpoints{point}{points}
	\qformat{\textbf{\large Question \thequestion}\quad\textbf{\large(\thepoints)}\hfill\rule[-3ex]{0ex}{3ex}}
}

%%%%%%%%%%%%%%%%%%%%%%%%%%%%%%%%%%%%%%%%%%%%%%%%%%%%%%%%%%%%%%%%%%%%%%%
%% Global Formatting Settings
%%%%%%%%%%%%%%%%%%%%%%%%%%%%%%%%%%%%%%%%%%%%%%%%%%%%%%%%%%%%%%%%%%%%%%%
\setlength{\parindent}{0cm}
\setlength{\parskip}{2ex}
\setlength{\textwidth}{160mm} \setlength{\topmargin}{-6mm}
\setlength{\textheight}{230mm}

\setlength{\oddsidemargin}{0mm} \setlength{\evensidemargin}{0mm}

\clubpenalty=1000000 \widowpenalty=1000000


%%%%%%%%%%%%%%%%%%%%%%%%%%%%%%%%%%%%%%%%%%%%%%%%%%%%%%%%%%%%%%%%%%%%%%%
%% Headers & Footers
%%%%%%%%%%%%%%%%%%%%%%%%%%%%%%%%%%%%%%%%%%%%%%%%%%%%%%%%%%%%%%%%%%%%%%%

% page style for front page
\fancypagestyle{frontpage}{
	\fancyhf{}
	
	%remove top line
	\renewcommand{\headrulewidth}{0pt}

	\fancyfoot[L]{
		\examID~\examTitleShort
	}
	\fancyfoot[C]{
		\examPeriod
	}
	\fancyfoot[R]{
		\iftoggle{isGerman}{
  			Seite \thepage\ von \pageref{LastPage}
		}{
			Page \thepage\ of \pageref{LastPage}
		}
	}
}

% page style for all other pages
\fancypagestyle{otherpages}{
	\fancyhf{}

	\fancyhead[L]{
		\iftoggle{isGerman}{
  			Name: \hspace{7cm} Matrikelnummer:
		}{
			Name: \hspace{7cm} Matriculation Number:
		}
	}

	\fancyfoot[L]{
		\examID~\examTitleShort
	}
	\fancyfoot[C]{
		\examPeriod
	}
	\fancyfoot[R]{
		\iftoggle{isGerman}{
  			Seite \thepage\ von \pageref{LastPage}
		}{
			Page \thepage\ of \pageref{LastPage}
		}
	}
}


%%%%%%%%%%%%%%%%%%%%%%%%%%%%%%%%%%%%%%%%%%%%%%%%%%%%%%%%%%%%%%%%%%%%%%%
%% Additional Commands
%%%%%%%%%%%%%%%%%%%%%%%%%%%%%%%%%%%%%%%%%%%%%%%%%%%%%%%%%%%%%%%%%%%%%%%

% define todo command
\definecolor{darkred}{rgb}{.6,0,0}
\newcommand{\todo}[1]{{\bf \color{darkred}TODO: [{#1}]}}



% TODO: check and clean following packages

%
% AMS-Operatoren
%
\DeclareMathOperator{\sign}{sign}

%
% Punkte auf der Nebendiagonale
%
\newcommand{\rdots}{\mathinner{
  \mkern1mu\raise1pt\hbox{.}
  \mkern2mu\raise4pt\hbox{.}
  \mkern2mu\raise7pt\vbox{\kern7pt\hbox{.}}\mkern1mu}}

%
%Vektoren/Matizen fett
%
\renewcommand{\vec}[1]{\boldsymbol{#1}}

%ermˆglicht fette griechische Buchstaben
\newcommand{\MB}[1]{{\mbox{\mathversion{bold}$#1$}}}

%
%Matrix (runde Klammer)
%
\newcommand{\mat}[1]{\begin{pmatrix}#1\end{pmatrix}}

%
%Darstellung von Hinweisen
%
\newcommand{\Hinweis}{\textbf{Hinweis:~}}
\newcommand{\Hint}{\textbf{Hint:~}}
\newcommand{\Annotation}{\textbf{Annotation:~}}

%
%Partielle Ableitung mit Potenz.
%Nutzung: \parfr[potenz(optional)]{z‰hler}{nenner}
%
\newcommand{\parfr}[3][\empty]{
	\ifthenelse{\equal{#1}{\empty}}
	{\frac{\partial {#2}}{\partial {#3}}}
	{\frac{\partial^{#1} {#2}}{\partial {#3}^{#1}}}
	}
	
%
%Normale Ableitung mit Potenz
%Nutzung: \dfr[potenz(optional)]{z‰hler}{nenner}
%
\newcommand{\dfr}[3][\empty]{
	\ifthenelse{\equal{#1}{\empty}}
	{\frac{\text{d} {#2}}{\text{d} {#3}}}
	{\frac{\text{d}^{#1} {#2}}{\text{d} {#3}^{#1}}}
	}

%
%Zahl in einem Kreis, geht im Text und auch in eine Formel
%
\newcommand{\mycirc}[1]{\mbox{\textcircled{\footnotesize #1}}}

%
%Bepunktung, geht im Text und auch in eine Formel
%
\newcommand{\Punkte}[1]{\text{~\textbf{($\boldsymbol{#1}$~P)}}}


\addto\captionsenglish{\renewcommand{\figurename}{Fig.}}


